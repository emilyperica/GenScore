% THIS DOCUMENT IS FOLLOWS THE VOLERE TEMPLATE BY Suzanne Robertson and James Robertson
% ONLY THE SECTION HEADINGS ARE PROVIDED
%
% Initial draft from https://github.com/Dieblich/volere
%
% Risks are removed because they are covered by the Hazard Analysis
\documentclass[12pt]{article}

\usepackage{booktabs}
\usepackage{tabularx}
\usepackage{hyperref}
\usepackage{longtable}
\usepackage{enumitem}
\hypersetup{
    bookmarks=true,         % show bookmarks bar?
      colorlinks=true,      % false: boxed links; true: colored links
    linkcolor=red,          % color of internal links (change box color with linkbordercolor)
    citecolor=green,        % color of links to bibliography
    filecolor=magenta,      % color of file links
    urlcolor=cyan           % color of external links
}

\newcommand{\lips}{\textit{Insert your content here.}}

\input{../Comments}
%% Common Parts

\newcommand{\progname}{ScoreGen} % PUT YOUR PROGRAM NAME HERE
\newcommand{\authname}{Team \#7, Tune Goons
\\ Emily Perica
\\ Ian Algenio
\\ Jackson Lippert
\\ Mark Kogan} % AUTHOR NAMES                  

\usepackage{hyperref}
    \hypersetup{colorlinks=true, linkcolor=blue, citecolor=blue, filecolor=blue,
                urlcolor=blue, unicode=false}
    \urlstyle{same}
                                


\begin{document}

\title{Software Requirements Specification for \progname: subtitle describing software} 
\author{\authname}
\date{\today}
	
\maketitle

~\newpage

\pagenumbering{roman}

\tableofcontents

~\newpage

\section*{Revision History}

\begin{tabularx}{\textwidth}{p{3cm}p{2cm}X}
\toprule {\textbf{Date}} & {\textbf{Version}} & {\textbf{Notes}}\\
\midrule
Date 1 & 1.0 & Notes\\
Date 2 & 1.1 & Notes\\
\bottomrule
\end{tabularx}

~\\

~\newpage
\section{Purpose of the Project}
\subsection{User Business}
\lips
\subsection{Goals of the Project}
\lips
\section{Stakeholders}
\subsection{Client}
\lips
\subsection{Customer}
\lips
\subsection{Other Stakeholders}
\lips
\subsection{Hands-On Users of the Project}
\lips
\subsection{Personas}
\lips
\subsection{Priorities Assigned to Users}
\lips
\subsection{User Participation}
\lips
\subsection{Maintenance Users and Service Technicians}
\lips

\section{Mandated Constraints}
\subsection{Solution Constraints}
\lips
\subsection{Implementation Environment of the Current System}
\lips
\subsection{Partner or Collaborative Applications}
\lips
\subsection{Off-the-Shelf Software}
\lips
\subsection{Anticipated Workplace Environment}
\lips
\subsection{Schedule Constraints}
\lips
\subsection{Budget Constraints}
\lips
\subsection{Enterprise Constraints}
\lips

\section{Naming Conventions and Terminology}
\subsection{Glossary of All Terms, Including Acronyms, Used by Stakeholders
involved in the Project}
\lips

\section{Relevant Facts And Assumptions}
\subsection{Relevant Facts}
\lips
\subsection{Business Rules}
\lips
\subsection{Assumptions}
\lips

\section{The Scope of the Work}
\subsection{The Current Situation}
\lips
\subsection{The Context of the Work}
\lips
\subsection{Work Partitioning}
\lips
\subsection{Specifying a Business Use Case (BUC)}
\lips

\section{Business Data Model and Data Dictionary}
\subsection{Business Data Model}
\lips
\subsection{Data Dictionary}
\lips

\section{The Scope of the Product}
\subsection{Product Boundary}
Input: ScoreGen will be designed to take in and process music data which can be converted to a MIDI or musicXML file representation. The audio should not have excessive noise, or other degradation of quality which could impact the ability to clearly distinguish notes. The input will also be intended to be mono or polyphonic with a singular instrument. The input volume should remain consistent and sufficiently loud, and the user microphone should be correctly calibrated to ensure accuracy by avoiding pitch drift.
\\ \\
Output: The resulting music file will aim to capture the pitch and length of individual notes or chords, but will not capture accents, dynamics, or tempo markings, as well as other advanced elements. If the system detects that a note cannot be reliably transcribed (due to unclear input), it will leave that note blank on the sheet and prompt the user to review the section manually.
\subsection{Product Use Case}
\section*{UC1: Record and Transcribe}
\textbf{Description:} Musician plays an instrument, and the system transcribes it into sheet music in real-time. \\
\textbf{Actors:} Musician, Desktop App \\
\textbf{Preconditions:} Musician has a working microphone and is playing a supported instrument. \\
\textbf{Basic Flow:}
\begin{enumerate}
    \item Musician plays the instrument.
    \item The system listens through the mic.
    \item Sheet music is generated.
\end{enumerate}
\textbf{Alternate Flow:}
\begin{itemize}
    \item If the note is unclear, the system skips it and warns the user.
\end{itemize}
\textbf{Postconditions:} Sheet music is generated and displayed to the user. 

\section*{UC2: Edit Transcription}
\textbf{Description:} Musician reviews and edits the generated sheet music if there are missing or incorrect notes. \\
\textbf{Actors:} Musician, Desktop App \\
\textbf{Preconditions:} Sheet music has already been generated from a recording. \\
\textbf{Basic Flow:}
\begin{enumerate}
    \item Musician views the transcription.
    \item Edits or corrects notes as needed.
    \item Saves the updated sheet.
\end{enumerate}
\textbf{Alternate Flow:}
\begin{itemize}
    \item Musician manually inserts missed or unclear notes.
    \item Musician deletes incorrect notes.
\end{itemize}
\textbf{Postconditions:} Musician has an edited, correct version of the sheet music.

\section*{UC3: Save and Export}
\textbf{Description:} Save the sheet music in a chosen format (e.g., PDF, MIDI, musicXML). \\
\textbf{Actors:} Musician, Desktop App \\
\textbf{Preconditions:} Sheet music is completed and reviewed. \\
\textbf{Basic Flow:}
\begin{enumerate}
    \item Musician chooses the save option.
    \item Selects a file format (PDF, MIDI, musicXML).
    \item System exports the file.
\end{enumerate}
\textbf{Alternate Flow:}
\begin{itemize}
    \item If export fails, the system provides an error message and suggests retrying.
\end{itemize}
\textbf{Postconditions:} The sheet music file is saved in the selected format.

\section*{UC4: Instrument Setup}
\textbf{Description:} Select instrument type before recording (e.g., piano, guitar). \\
\textbf{Actors:} Musician, Desktop App \\
\textbf{Preconditions:} Desktop app is open, and the user is ready to begin recording. \\
\textbf{Basic Flow:}
\begin{enumerate}
    \item Musician selects the instrument.
    \item System optimizes transcription settings for the chosen instrument.
\end{enumerate}
\textbf{Alternate Flow:}
\begin{itemize}
    \item If no instrument is selected, default settings for piano are used.
\end{itemize}
\textbf{Postconditions:} System is ready to accurately transcribe the chosen instrument.

\section*{UC5: Error Warning}
\textbf{Description:} System warns if it cannot hear or recognize a note during the recording process. \\
\textbf{Actors:} Musician, Desktop App \\
\textbf{Preconditions:} Musician is actively recording. \\
\textbf{Basic Flow:}
\begin{enumerate}
    \item System detects a missing or unclear note.
    \item A warning is displayed to the user.
\end{enumerate}
\textbf{Alternate Flow:}
\begin{itemize}
    \item The system may suggest trying again or inserting the note manually post-recording.
\end{itemize}
\textbf{Postconditions:} Musician is informed about missing notes and can take corrective action.

\section*{UC6: Playback of Transcription}
\textbf{Description:} Musician can play back the recorded notes in the desktop app to review transcription accuracy. \\
\textbf{Actors:} Musician, Desktop App \\
\textbf{Preconditions:} Transcription has been generated from a recording. \\
\textbf{Basic Flow:}
\begin{enumerate}
    \item Musician selects the playback option.
    \item System plays the transcribed notes.
\end{enumerate}
\textbf{Alternate Flow:}
\begin{itemize}
    \item Musician can pause or stop playback to make corrections.
\end{itemize}
\textbf{Postconditions:} Musician reviews the transcription by listening to the played-back notes.




\subsection{Individual Product Use Cases (PUC's)}
\subsection*{PUC1: Record Instrument and Transcribe Notes}
\begin{description}[style=nextline]
    \item[Description:] The user records a live performance on an instrument using their microphone, and the system transcribes the notes into sheet music.
    \item[Primary Actor:] Musician
    \item[Trigger:] The user presses the "Record" button.
    \item[Preconditions:]
    \begin{itemize}
        \item User has selected an instrument (e.g., piano, guitar).
        \item User has a working microphone connected to the desktop app.
    \end{itemize}
    \item[Main Success Scenario:]
    \begin{enumerate}
        \item User begins playing the instrument.
        \item System listens to the instrument through the microphone.
        \item Sheet music is displayed on the screen after the user finishes playing.
    \end{enumerate}
    \item[Exceptions:]
    \begin{itemize}
        \item If the system cannot detect a note, it skips the note and warns the user with a notification.
    \end{itemize}
\end{description}

\subsection*{PUC2: Edit Transcription}
\begin{description}[style=nextline]
    \item[Description:] The user reviews the transcribed sheet music and edits any incorrect or missing notes.
    \item[Primary Actor:] Musician
    \item[Trigger:] The user clicks on the "Edit" button after the sheet music is generated.
    \item[Preconditions:]
    \begin{itemize}
        \item The sheet music has already been transcribed from a recording.
    \end{itemize}
    \item[Main Success Scenario:]
    \begin{enumerate}
        \item The user clicks on a note or measure to edit.
        \item The user adjusts the pitch, duration, or deletes notes.
        \item The user saves the changes.
        \item The updated sheet music is displayed.
    \end{enumerate}
    \item[Exceptions:]
    \begin{itemize}
        \item If the user makes a mistake, they can undo the last change.
    \end{itemize}
\end{description}

\subsection*{PUC3: Save and Export Sheet Music}
\begin{description}[style=nextline]
    \item[Description:] The user saves the transcribed sheet music and exports it in the desired format (PDF, MIDI, musicXML).
    \item[Primary Actor:] Musician
    \item[Trigger:] The user clicks the "Save" or "Export" button.
    \item[Preconditions:]
    \begin{itemize}
        \item The transcription is completed, and the user has reviewed it.
    \end{itemize}
    \item[Main Success Scenario:]
    \begin{enumerate}
        \item The user selects a file format (PDF, MID, musicXMLI).
        \item The system generates a file in the selected format.
        \item The file is saved to the user’s chosen directory.
    \end{enumerate}
    \item[Exceptions:]
    \begin{itemize}
        \item If the export fails due to file system issues, the user is prompted to retry or select another directory.
    \end{itemize}
\end{description}

\subsection*{PUC4: Playback Transcribed Notes}
\begin{description}[style=nextline]
    \item[Description:] The user listens to the transcription to review the accuracy of the notes.
    \item[Primary Actor:] Musician
    \item[Trigger:] The user presses the "Play" button.
    \item[Preconditions:]
    \begin{itemize}
        \item The sheet music has been generated from the recording.
    \end{itemize}
    \item[Main Success Scenario:]
    \begin{enumerate}
        \item The system generates and plays audio of the transcribed notes in sequence.
        \item The user listens and evaluates the accuracy of the transcription.
    \end{enumerate}
    \item[Exceptions:]
    \begin{itemize}
        \item If playback fails, the user is prompted with an error message and retry option.
    \end{itemize}
\end{description}

\subsection*{PUC5: Receive Error Warnings for Missing Notes}
\begin{description}[style=nextline]
    \item[Description:] The user is notified if the system cannot detect or accurately transcribe certain notes during recording.
    \item[Primary Actor:] Musician
    \item[Trigger:] The system fails to detect a note.
    \item[Preconditions:]
    \begin{itemize}
        \item User is actively recording their instrument.
    \end{itemize}
    \item[Main Success Scenario:]
    \begin{enumerate}
        \item The system detects a missing note during recording.
        \item A warning message is displayed to the user regarding the missing note.
    \end{enumerate}
    \item[Exceptions:]
    \begin{itemize}
        \item The system suggests the user retry playing the note or to manually insert it later.
    \end{itemize}
\end{description}


\section{Functional Requirements}
\subsection{Functional Requirements}
\lips

\section{Look and Feel Requirements}
\subsection{Appearance Requirements}
\lips
\subsection{Style Requirements}
\lips

\section{Usability and Humanity Requirements}
\subsection{Ease of Use Requirements}
\lips
\subsection{Personalization and Internationalization Requirements}
\lips
\subsection{Learning Requirements}
\lips
\subsection{Understandability and Politeness Requirements}
\lips
\subsection{Accessibility Requirements}
\lips

\section{Performance Requirements}
\subsection{Speed and Latency Requirements}
\lips
\subsection{Safety-Critical Requirements}
\lips
\subsection{Precision or Accuracy Requirements}
\lips
\subsection{Robustness or Fault-Tolerance Requirements}
\lips
\subsection{Capacity Requirements}
\lips
\subsection{Scalability or Extensibility Requirements}
\lips
\subsection{Longevity Requirements}
\lips

\section{Operational and Environmental Requirements}
\subsection{Expected Physical Environment}
\lips
\subsection{Wider Environment Requirements}
\lips
\subsection{Requirements for Interfacing with Adjacent Systems}
\lips
\subsection{Productization Requirements}
\lips
\subsection{Release Requirements}
\lips

\section{Maintainability and Support Requirements}
\subsection{Maintenance Requirements}
\lips
\subsection{Supportability Requirements}
\lips
\subsection{Adaptability Requirements}
\lips

\section{Security Requirements}
\subsection{Access Requirements}
\lips
\subsection{Integrity Requirements}
\lips
\subsection{Privacy Requirements}
\lips
\subsection{Audit Requirements}
\lips
\subsection{Immunity Requirements}
\lips

\section{Cultural Requirements}
\subsection{Cultural Requirements}
\lips

\section{Compliance Requirements}
\subsection{Legal Requirements}
\lips
\subsection{Standards Compliance Requirements}
\lips

\section{Open Issues}

\textbf{Handling Complex Rhythms and Non-Monophonic Instruments:} Scoregen may struggle to accurately transcribe complex rhythms (e.g., syncopation, irregular time signatures) or multiple instruments played simultaneously. \\
\textbf{Real-Time Processing Latency:  }Potential issue if some near real-time note generation and display system is implemented. There could be a delay between when the instrument is played and when the notes appear on the screen, disrupting the real-time transcription experience. \\
\textbf{Inconsistent Detection of Chords:}  The system may have difficulty accurately detecting and transcribing complex chords or overlapping harmonic frequencies, leading to errors.
\textbf{Difficulty with Non-Standard Tuning or Instruments:} Musicians using non-standard tuning or less common instruments may experience inaccurate note detection or transcription errors. \\
\textbf{Microphone and Equipment Dependency:} The quality and type of microphone used can significantly affect transcription accuracy, leading to more errors for users with lower-quality equipment. \\
\textbf{Legal and Licensing Concerns:} There may be intellectual property issues if users transcribe and share sheet music for copyrighted music without permission.

\section{Off-the-Shelf Solutions}
\subsection{Ready-Made Products}
\lips
\subsection{Reusable Components}
\lips
\subsection{Products That Can Be Copied}
\lips

\section{New Problems}
\subsection{Effects on the Current Environment}
\lips
\subsection{Effects on the Installed Systems}
\lips
\subsection{Potential User Problems}
\lips
\subsection{Limitations in the Anticipated Implementation Environment That May
Inhibit the New Product}
\lips
\subsection{Follow-Up Problems}
\lips

\section{Tasks}
\subsection{Project Planning}
\lips
\subsection{Planning of the Development Phases}
\lips

\section{Migration to the New Product}
\subsection{Requirements for Migration to the New Product}
\lips
\subsection{Data That Has to be Modified or Translated for the New System}
\lips

\section{Costs}
\textbf{Mid-Range Microphones: \$100 - \$500:} \\
High quality Microphones are essential to capture audio by converting sound waves into electrical signals. If necessary, it may be beneficial for the project to acquire higher quality microphones for testing. \\ \\
\textbf{Storage for Large Data Files:} \\
Cloud or local storage could be an additional potential cost, as features like account management, stored sheet music, or other data could require subscription of a cloud database server (\$100 - \$500 per year), or purchasing more memory (\$50 - \$400).

\section{User Documentation and Training}
\subsection{User Documentation Requirements}
\textbf{Installation and Setup Guide:} \\ This documentation should provide clear instructions for downloading, installing, and configuring the software, including system requirements and initial setup steps to ensure users can easily get the application running on their devices.
\\ 
\textbf{Quick Start Guide:}\\This section should offer a streamlined overview of the essential features and workflow of the application, allowing users to quickly record their first piece of music and generate sheet music with minimal effort.\\
\textbf{Tutorials and Walkthroughs: } \\Developers should document step-by-step guides for various user scenarios, from basic recording to advanced features, ensuring users can understand how to effectively use all functionalities of the application. \\
\textbf{Version History and Updates: }\\ This documentation should detail changes made in each version of the software, including new features, improvements, and bug fixes, helping users stay informed about updates and how they might impact their use of the application. \\
\textbf{Feedback and Support Channels: } \\Clear documentation should outline how users can provide feedback or seek support, including contact information for customer service or a support form, enabling users to get assistance with any issues they encounter. \\
\textbf{Glossary of Terms:}\\ This section should define key technical and musical terminology used throughout the software, helping non-expert users understand the language and concepts related to music transcription and audio processing.
\subsection{Training Requirements}
To ensure sufficient resources for product use, the development team will provide: \\
\textbf{Tutorial Videos:} \\ Develop a series of engaging tutorial videos that visually demonstrate how to use the software, covering basic and advanced features, to cater to various learning styles and help users quickly grasp the application’s functionalities. \\
\textbf{Feedback Mechanism:}\\ Establish a system for collecting user feedback on the training materials and sessions, enabling continuous improvement of the training content and methods based on user experiences and suggestions.


\section{Waiting Room}
\lips

\section{Ideas for Solution}
\lips

\newpage{}
\section*{Appendix --- Reflection}

\input{../Reflection.tex}

\begin{enumerate}
  \item What went well while writing this deliverable? 
  \item What pain points did you experience during this deliverable, and how did
  you resolve them?
  \item How many of your requirements were inspired by speaking to your
  client(s) or their proxies (e.g. your peers, stakeholders, potential users)?
  \item Which of the courses you have taken, or are currently taking, will help
  your team to be successful with your capstone project.
  \item What knowledge and skills will the team collectively need to acquire to
  successfully complete this capstone project?  Examples of possible knowledge
  to acquire include domain specific knowledge from the domain of your
  application, or software engineering knowledge, mechatronics knowledge or
  computer science knowledge.  Skills may be related to technology, or writing,
  or presentation, or team management, etc.  You should look to identify at
  least one item for each team member.
  \item For each of the knowledge areas and skills identified in the previous
  question, what are at least two approaches to acquiring the knowledge or
  mastering the skill?  Of the identified approaches, which will each team
  member pursue, and why did they make this choice?
\end{enumerate}


\end{document}