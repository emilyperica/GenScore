\documentclass{article}

\usepackage{booktabs}
\usepackage{tabularx}
\usepackage{longtable}

\title{Development Plan\\\progname}

\author{\authname}

\date{}

\input{../Comments}
%% Common Parts

\newcommand{\progname}{ScoreGen} % PUT YOUR PROGRAM NAME HERE
\newcommand{\authname}{Team \#7, Tune Goons
\\ Emily Perica
\\ Ian Algenio
\\ Jackson Lippert
\\ Mark Kogan} % AUTHOR NAMES                  

\usepackage{hyperref}
    \hypersetup{colorlinks=true, linkcolor=blue, citecolor=blue, filecolor=blue,
                urlcolor=blue, unicode=false}
    \urlstyle{same}
                                


\begin{document}

\maketitle

\begin{table}[hp]
\caption{Revision History} \label{TblRevisionHistory}
\begin{tabularx}{\textwidth}{llX}
\toprule
\textbf{Date} & \textbf{Developer(s)} & \textbf{Change}\\
\midrule
Date1 & Name(s) & Description of changes\\
Date2 & Name(s) & Description of changes\\
... & ... & ...\\
\bottomrule
\end{tabularx}
\end{table}

\newpage{}

\wss{Put your introductory blurb here.  Often the blurb is a brief roadmap of
what is contained in the report.}

\wss{Additional information on the development plan can be found in the
\href{https://gitlab.cas.mcmaster.ca/courses/capstone/-/blob/main/Lectures/L02b_POCAndDevPlan/POCAndDevPlan.pdf?ref_type=heads}
{lecture slides}.}

\section{Confidential Information?}

\wss{State whether your project has confidential information from industry, or
not.  If there is confidential information, point to the agreement you have in
place.}

\wss{For most teams this section will just state that there is no confidential
information to protect.}
\section{IP to Protect}

\wss{State whether there is IP to protect.  If there is, point to the agreement.
All students who are working on a project that requires an IP agreement are also
required to sign the ``Intellectual Property Guide Acknowledgement.''}

\section{Copyright License}

\wss{What copyright license is your team adopting.  Point to the license in your
repo.}

\section{Team Meeting Plan}
\subsection{General Team Meetings}
  \textbf{Meeting Chair}: Jackson Lippert.\\
  \textbf{Frequency}: Weekly.\\
  \textbf{Format}: In-person.\\
  \textbf{Location}: McMaster’s main campus.
    \begin{itemize}
        \item Particular location may vary, will be discussed prior to the meeting time.
    \end{itemize}
  \textbf{Time (primary)}: Monday, 2:30pm-4:20pm (the designated tutorial time slot for SFWRENG 4G06A/B).\\
  \textbf{Time (secondary)}: Monday, 12:30pm-1:30pm.
  \begin{itemize}
      \item The secondary time slot will be used if the primary slot is occupied by synchronous tutorials or at the discretion of team members.
  \end{itemize}
  \textbf{Structure}:
  \clearpage
  \begingroup
    \renewcommand{\arraystretch}{1.25}
    \begin{longtable}{|p{\textwidth}|}
      \hline
      \textbf{\textbf{AGENDA ITEMS (GENERAL MEETING)}} \\
      \hline
      \begin{enumerate}
        \itemsep0em
        \item Discuss/review team member updates.
        \begin{enumerate}
          \itemsep0em
          \item Recent changes made.
          \item Problems encountered.
          \item Status of previous meeting action items.
        \end{enumerate}
        \item Go over new business.
        \begin{enumerate}
          \itemsep0em
          \item Discuss future deliverables/tasks to be done.
          \item Define individual and/or group next steps.
        \end{enumerate}
        \item Assign action items.
      \end{enumerate} \\
      \hline
    \end{longtable}
  \endgroup

\subsection{Supervisor/Industry Advisor Meetings}
  \textbf{Frequency}: Biweekly.\\
  \textbf{Format}: In-person, virtual if necessary.\\
  \textbf{Location}: Dr. Martin v. Mohrenschildt’s campus office.\\
  \textbf{Time}: By appointment.\\
  \textbf{Structure}:
  \clearpage
  \begingroup
    \renewcommand{\arraystretch}{1.25}
    \begin{longtable}{|p{\textwidth}|}
      \hline
      \textbf{\textbf{AGENDA ITEMS (SUPERVISOR MEETING)}} \\
      \hline
      \begin{enumerate}
        \itemsep0em
        \item Discuss/review team updates.
        \begin{enumerate}
          \itemsep0em
          \item Project milestones reached.
          \item Status of previous meeting action items.
        \end{enumerate}
        \item Troubleshoot recent issues that require supervisor-specific knowledge.
        \item Determine high-level plan for project progression.
        \begin{enumerate}
          \itemsep0em
          \item Discuss problems the team expects to encounter in the near future.
          \item Identify goals the team wishes to achieve by the next meeting with the supervisor.
        \end{enumerate}
        \item Assign action items (if applicable).
        \item Discuss details of next supervisor meeting.
        \begin{enumerate}
          \itemsep0em
          \item Date \& time.
          \item Location.
        \end{enumerate}
      \end{enumerate} \\
      \hline
    \end{longtable}
  \endgroup
\textbf{Note}: All fields are subject to change depending on the supervisor’s availability or the team’s need to consult the supervisor.

\section{Team Communication Plan}
\subsection{General Purpose Communication}
The primary mode of communication for general purposes will be through Apple iMessage SMS/MMS in the team group chat.
In the event that a team member does not respond via iMessage, the secondary mode of communication will be used. This will 
be through an Instagram direct messages group chat. The primary and secondary communication channels were chosen due to their 
practicality, team members will almost certainly have their smartphone on hand, which they can use for both iMessage and Instagram 
direct messages.
A tertiary form of communication through Microsoft Teams direct messaging and/or video calls. This will be used in the event a team 
member does not respond within a reasonable time frame to primary and secondary communication methods. It will also be the main 
application through which virtual meetings will take place (both general and supervisor meetings) as it has screen sharing capabilities 
and is familiar to all team members.
\subsection{Implementation and Task Assignment}
GitHub Projects will be the primary tool used for project management related communication. A kanban board will be collectively
maintained by the team for task assignment, progress status, meeting bookkeeping, and item reviews. 
GitHub Issues will be used in tandem with the kanban board. As such, every opened issue will be required to also 
appear on the kanban board. In addition to project tasks and to-dos, issues will also be opened for every team meeting and 
every SFWRENG 4G06A/B lecture to track attendance and team contribution. There are five categories of issues that can be 
opened by the team, each having their own template: 
\begin{itemize}
  \item Lecture
  \item Meeting
  \begin{itemize}
    \item General team meetings
    \item Supervisor meetings
    \item TA meetings
  \end{itemize}
  \item Peer Review
\end{itemize}
The use of these templates standardize issue creation and streamlines project progression.


\section{Team Member Roles}
All listed team member roles are tenative and subject to change at the discretion of the team. Should role changes occur
team members should follow the process outlined below: 
\begin{enumerate}
  \item Contact all team members that are relevant to the role change.
  \item Discuss proposed changes and ensure each team member understands their new responsibilies.
  \item Notify uncontacted team members of the change and/or the overtaking of responsibilities. 
\end{enumerate}
\subsection{Project Manager}
\textbf{Assignee:} Emily Perica\\
The project manager directs the team towards successful completion of the project. 
They will be responsible for coordinating the progression by creating meeting agendas for the meeting chairperson 
to follow. They will also maintain the Kanban board and create and assign GitHub issues that concern the entire team.
\subsection{Meeting Chair}
\textbf{Assignee(s):} Jackson Lippert\\
The meeting chair is responsible for driving team meetings towards productivity and efficiency. 
This will be done by guiding and directing meetings to minimize off-topic conversation. They are required to ensure all 
meeting agenda items are addressed in a timely manner. Finally, they will facilitate constructive, collaborative
discussions during meetings.
\subsection{Meeting Minutes Recorder*}
\textbf{Assignee(s):} Ian Algenio\\
The primary responsibility of this role is to document and record team meetings. They will also track team member attendance
and address associated issues. They will record meeting discussions, issues, topics, action items, etc.
\subsection{Subject Matter Expert*}
\textbf{Assignee(s):} Mark Kogan\\
The subject matter expert is given the responsibility of maintaining a neat and organized record of source materials used 
during the lifetime of the project. This may include academic articles, open-source repositories, external libraries used, etc. 
They will also create any necessary citations for project documentation elements.
\subsection{Software Developer*}
\textbf{Assignee(s):} ALL MEMBERS\\
All team members will take on the responsibilites of this role. As developers they are responsible for designing,
testing, coding, and maintaining the software components of the project.
\subsection{Quality Assurance and Testing Specialist*}
\textbf{Assignee(s):} ALL MEMBERS\\
This role ensures the software meets stakeholder needs and quality standards. They will develop and
execute test plans and test cases, measure code coverage, and report defects in the software. Additionally,
they will verify the software meets functional and non-functional requirements such as performance, security, usability, etc.

\subsection{Music Domain Expert}
\textbf{Assignee(s):} Emily Perica\\
As the expert in the music theory domain, this person will address any confusion regarding general music theory such as terminology. They will
share their prior knowledge and experience with music whether that's playing an instrument or clarifying details about the domain.\\

\textbf{Note:} Any team member roles marked with an asterisk (*) are able to be assigned to multiple people, 
have overlapping responsibilities with other roles, and/or have responsibilities are likely to be shared amongst the enrire team.

\section{Workflow Plan}

\textbf{Issue Management Workflows}\\
When creating an issue in the Capstone Kanban Board, there are two possible workflows:
\begin{enumerate}
  \item Meeting issue is created (using the relevant template) prior to a scheduled meeting.  
  \begin{enumerate}
    \item Issue is given the 'meeting' label and assigned the 'Meeting Minutes' status.
    \item The notetaker (see section 6) will add a comment to the issue, outlining the meeting minutes and action items of each attendee.
    \item Issue is closed once all action items have been addressed.
  \end{enumerate}
  \item Project work issue is created using a blank issue.
  \begin{enumerate}
    \item Issue begins in 'To Do'
    \item One or more appropriate labels are assigned - ‘documentation’, ‘deliverable’, ‘enhancement’, ‘bug’, and potentially others as the need arises.
    \item If the issue does not have the ‘deliverable’ label AND is not time sensitive, it will be moved to ‘Backlog’.
    \begin{enumerate}
      \item The issue will move out of 'Backlog' and into 'To Do' if the issue becomes time sensitive or the team has the time/resources to take it on.
    \end{enumerate}
    \item The issue moves to ‘In Progress’ once it is assigned and being worked on. It may be assigned to more than one person, depending on the scope of the issue.
    \begin{enumerate}
      \item At this point, an issue pertaining to the codebase will branch to the Git Workflow (below).
    \end{enumerate}
    \item The issue moves to ‘In Review’ once a PR has been made and approval is requested from the team.
    \item The issue moves to ‘Done’ and can be closed once all associated tasks have been completed, including all related PRs being merged/closed.
  \end{enumerate}
\end{enumerate}

In addition to the above usage of the Kanban board, all course deliverables are documented as milestones in the repository and will have issues assigned as necessary. Each milestone has a due date and a description of the tasks to be completed.

The Project Manager (see section 6 for team member roles) is responsible for making issues and keeping the project on track. As well, they will run monthly backlog grooming sessions with the team to ensure any stale issues are taken care of. \\

\noindent
\textbf{Git/GitHub Workflow}\\
A new developer on the project will begin at step 1; otherwise step 1 can be skipped:
\begin{enumerate}
  \item Developer may choose to fork or clone the repo. A fork is recommended so that contributors can display the project directly in their GitHub profile.
  \item Main is a protected branch - any commits must be made in a separate branch before being pushed to a PR. There is no specific naming scheme for branches.
  \item Commit and PR names will both start with ‘Issue \#: ’. This will be enforced through a GitHub Action.
  \begin{enumerate}
    \item Each PR will only address a single issue, but an issue may be spread out across multiple PRs.
    \item Squash commits are enforced to ensure there is a single commit per PR.
  \end{enumerate}
  \item At least one reviewer must approve the PR before it can be merged into main.\\
\end{enumerate}


\noindent
\textbf{CI/CD Workflow}\\
Actions on Push/Pull Request:
\begin{itemize}
  \item LaTeX to PDF converter
  \item Code packaged as an executable
  \item Linting
  \item Unit tests
  \item PR naming convention enforcer
\end{itemize}

Note that a PR may not be merged if any failures are present in the pipeline.

\section{Project Decomposition and Scheduling}

\begin{center}
  \begin{tabular}{ |c|c| } 
    \hline
      \textbf{Due Date} & \textbf{Deliverable} \\
    \hline
      Sept 23/24 & Problem Statement, POC Plan, Development Plan \\
    \hline
      Oct 9/24 & Requirements Document Revision 0 \\
    \hline
      Oct 23/24 & Hazard Analysis \\
    \hline
      Nov 1/24 & V\&V Plan Revision 0 \\
    \hline
      Nov 11/24* & Proof of Concept Demonstration \\
    \hline
      Jan 15/25 & Design Document Revision 0 \\
    \hline
      Feb 3/25* & Revision 0 Demonstration \\
    \hline
      Mar 7/25 & V\&V Report Revision 0 \\
    \hline
      Mar 24/25* & Final Demonstration (Revision 1) \\
    \hline
      Apr \#\#/25 - TBD & Expo Demonstration \\
    \hline
      Apr 2/25 & Final Documentation (Revision 1) \\
    \hline
  \end{tabular}
\end{center}
*The demonstration should be prepared by this date, but this will not necessarily be the date of the demo itself.

Ideally, progress on a deliverable will begin at least 3 days before the previous deliverable is due (i.e., D2 work should begin on or before Sept 20). The above schedule will be enforced through the use of Milestones in the project repository, such that each deliverable (including its due date and all relevant information) is its own Milestone.
The only exception to this is the Expo Demonstration, which is not expected to require any specific issues.

Each deliverable will be split into issues based on dependencies and relevant tasks. For example, when writing this document a single issue consisted of sections 7 and 8. They both cover project management-related plans, so completing one section provides the writer with the needed knowledge to write the other section. An alternative proposed method was to create an issue for each section of the document, but the granularity of this strategy may make the document feel more disjointed to the reader.

\section{Proof of Concept Demonstration Plan}

What is the main risk, or risks, for the success of your project?  What will you
demonstrate during your proof of concept demonstration to convince yourself that
you will be able to overcome this risk?

\section{Expected Technology}

\subsection{Tabular Overview}
\begingroup
\renewcommand{\arraystretch}{1.25}
\begin{tabular}{|>{\raggedright}p{3cm}|>{\raggedright}p{4.25cm}|>{\raggedright\arraybackslash}p{6cm}|}
  \hline
  DOMAIN & TOOL/TECHNOLOGY & DESCRIPTION \\
  \hline
  Version Control & Git, GitHub & For version control and issue tracking respectively.\\
  \hline
  Project Management & GitHub Projects & Used for Kanban board, task assignment, backlogging, etc. \\
  \hline
  Back-End Development & C/C++, MATLAB & Possible specific languages, high performance and speed is required for
  signal processesing.\\
  \hline
  Sound File Conversion & libsndfile  & A specific library that was researched and will likely be used
  for sampled sound file format conversion.\\
  \hline
  Code Formatting & Linter(s) &  Specific tools are undetermined at this stage as it is language dependent. 
  To be used with CI workflows to format code.\\
  \hline
  Testing & MATLAB unittest/automation, CMake, Unity Test, etc.  & Specific tools are undetermined, likely a unit testing framework
  that does not transcend languages (i.e. language-specific).
  \\
  \hline
  AI/ML & N/A & Will use existing libraries (pytorch, tensorflow, etc.) given the time constraint of the project, however,
  the team aims to generate our own datasets for training.\\
  \hline
  Front-End Development & HTML, CSS, JS, React Native & Languages and existing libraries deemed useful for web-based
  GUI. A framework will likely be used, in particular React Native.\\
  \hline
  UI/UX & Figma, Canva & To design and effectively prototype UI/UX elements.\\
  \hline
  Hardware & Audio Interface & An external unit for I/O of audio signals to the host device. Possible to use the host computer's 
  built-in microphone as opposed to this. \\
  \hline
  Musical Instruments & Keyboard, guitar, ukelele, etc. & Used to generate input signals.\\
  \hline
\end{tabular}
\endgroup
\subsection{Continuous Integration Plans}
GitHub Actions (GHA) will be used for continuous integration. They will employ automated workflows for various unit and integration tests
as well as code formatting/linting. One specific workflow that will be created is .tex to .pdf format conversion to ease documentation editing.
GHA will also ideally be used to test code coverage metrics. One of the major components of the project is the signal processing algorithm.  
Depending on the implementation details, different metrics in addition to the basic ones such as statement coverage might be used. For example, the designed algorithm 
might benefit from condition or path coverage for pitch detection.

\section{Coding Standard}

\wss{What coding standard will you adopt?}

\newpage{}

\section*{Appendix --- Reflection}

\wss{Not required for CAS 741}

\input{../Reflection.tex}

\begin{enumerate}
    \item Why is it important to create a development plan prior to starting the
    project?
    \item In your opinion, what are the advantages and disadvantages of using
    CI/CD?
    \item What disagreements did your group have in this deliverable, if any,
    and how did you resolve them?
\end{enumerate}

\newpage{}

\section*{Appendix --- Team Charter}

\wss{borrows from
\href{https://engineering.up.edu/industry_partnerships/files/team-charter.pdf}
{University of Portland Team Charter}}

\subsection*{External Goals}

\wss{What are your team's external goals for this project? These are not the
goals related to the functionality or quality fo the project.  These are the
goals on what the team wishes to achieve with the project.  Potential goals are
to win a prize at the Capstone EXPO, or to have something to talk about in
interviews, or to get an A+, etc.}

\subsection*{Attendance}

\subsubsection*{Expectations}

\wss{What are your team's expectations regarding meeting attendance (being on
time, leaving early, missing meetings, etc.)?}

\subsubsection*{Acceptable Excuse}

\wss{What constitutes an acceptable excuse for missing a meeting or a deadline?
What types of excuses will not be considered acceptable?}

\subsubsection*{In Casef Emergency}

\wss{What process will team members follow if they have an emergency and cannot
attend a team meeting or complete their individual work promised for a team
deliverable?}

\subsection*{Accountability and Teamwork}

\subsubsection*{Quality} 

\wss{What are your team's expectations regarding the quality
of team members' preparation for team meetings and the quality of the
deliverables that members bring to the team?}

\subsubsection*{Attitude}

\wss{What are your team's expectations regarding team members' ideas,
interactions with the team, cooperation, attitudes, and anything else regarding
team member contributions?  Do you want to introduce a code of conduct?  Do you
want a conflict resolution plan?  Can adopt existing codes of conduct.}

\subsubsection*{Stay on Track}

\wss{What methods will be used to keep the team on track? How will your team
ensure that members contribute as expected to the team and that the team
performs as expected? How will your team reward members who do well and manage
members whose performance is below expectations?  What are the consequences for
someone not contributing their fair share?}

\wss{You may wish to use the project management metrics collected for the TA and
instructor for this.}

\wss{You can set target metrics for attendance, commits, etc.  What are the
consequences if someone doesn't hit their targets?  Do they need to bring the
coffee to the next team meeting?  Does the team need to make an appointment with
their TA, or the instructor?  Are there incentives for reaching targets early?}

\subsubsection*{Team Building}

\wss{How will you build team cohesion (fun time, group rituals, etc.)? }

\subsubsection*{Decision Making} 

\wss{How will you make decisions in your group? Consensus?  Vote? How will you
handle disagreements? }

\end{document}
