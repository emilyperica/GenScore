\documentclass{article}

\usepackage{booktabs}
\usepackage{tabularx}
\usepackage{longtable}

\title{Development Plan\\\progname}

\author{\authname}

\date{}

\input{../Comments}
%% Common Parts

\newcommand{\progname}{ScoreGen} % PUT YOUR PROGRAM NAME HERE
\newcommand{\authname}{Team \#7, Tune Goons
\\ Emily Perica
\\ Ian Algenio
\\ Jackson Lippert
\\ Mark Kogan} % AUTHOR NAMES                  

\usepackage{hyperref}
    \hypersetup{colorlinks=true, linkcolor=blue, citecolor=blue, filecolor=blue,
                urlcolor=blue, unicode=false}
    \urlstyle{same}
                                


\begin{document}

\maketitle

\begin{table}[hp]
\caption{Revision History} \label{TblRevisionHistory}
\begin{tabularx}{\textwidth}{llX}
\toprule
\textbf{Date} & \textbf{Developer(s)} & \textbf{Change}\\
\midrule
2024/09/24 & Mark, Ian, Emily, Jackson & Initial Revision\\
... & ... & ...\\
\bottomrule
\end{tabularx}
\end{table}

\newpage{}

The development plan contains our full overview of our goals and the subsequent methods we’ve chosen to accomplish them. 
It serves as a guide which will be referenced by each developer throughout the product cycle to create a standard of expected protocol, behavior, goals, and development practices. 
In this document, we have outlined how we will conduct team meetings, as well as the meeting frequency. Further outlined is communication guidelines, team member roles, development protocols, expected tech, deliverable timelines and our individual and group reflections. 
Overall this document shows how the group plans to develop the product through an initial proof of concept, followed by a minimal viable product and continuous development to improve the result until it satisfies our intended goals.

\section{Confidential Information?}

There is no confidential information needing protection for our project.

\section{IP to Protect}

There is no IP to protect for our project.

\section{Copyright License}

Our team will be adopting the \href{https://github.com/emilyperica/ScoreGen/blob/main/LICENSE}{MIT License}, which is standard for allowing usage without limitation
to use, copy, modify, merge, publish, distribute, sublicense, and/or sell
copies of the Software. This is due to the nature of our project, which is focused on learning and not for commercial use.


\section{Team Meeting Plan}
\subsection{General Team Meetings}
  \textbf{Meeting Chair}: Jackson Lippert.\\
  \textbf{Frequency}: Weekly.\\
  \textbf{Format}: In-person.\\
  \textbf{Location}: McMaster’s main campus.
    \begin{itemize}
        \item Particular location may vary, will be discussed prior to the meeting time.
    \end{itemize}
  \textbf{Time (primary)}: Monday, 2:30pm-4:20pm (the designated tutorial time slot for SFWRENG 4G06A/B).\\
  \textbf{Time (secondary)}: Monday, 12:30pm-1:30pm.
  \begin{itemize}
      \item The secondary time slot will be used if the primary slot is occupied by synchronous tutorials or at the discretion of team members.
  \end{itemize}
  \textbf{Structure}:
  \clearpage
  \begingroup
    \renewcommand{\arraystretch}{1.25}
    \begin{longtable}{|p{\textwidth}|}
      \hline
      \textbf{\textbf{AGENDA ITEMS (GENERAL MEETING)}} \\
      \hline
      \begin{enumerate}
        \itemsep0em
        \item Discuss/review team member updates.
        \begin{enumerate}
          \itemsep0em
          \item Recent changes made.
          \item Problems encountered.
          \item Status of previous meeting action items.
        \end{enumerate}
        \item Go over new business.
        \begin{enumerate}
          \itemsep0em
          \item Discuss future deliverables/tasks to be done.
          \item Define individual and/or group next steps.
        \end{enumerate}
        \item Assign action items.
      \end{enumerate} \\
      \hline
    \end{longtable}
  \endgroup

\subsection{Supervisor/Industry Advisor Meetings}
  \textbf{Frequency}: Biweekly.\\
  \textbf{Format}: In-person, virtual if necessary.\\
  \textbf{Location}: Dr. Martin v. Mohrenschildt’s campus office.\\
  \textbf{Time}: By appointment.\\
  \textbf{Structure}:
  \clearpage
  \begingroup
    \renewcommand{\arraystretch}{1.25}
    \begin{longtable}{|p{\textwidth}|}
      \hline
      \textbf{\textbf{AGENDA ITEMS (SUPERVISOR MEETING)}} \\
      \hline
      \begin{enumerate}
        \itemsep0em
        \item Discuss/review team updates.
        \begin{enumerate}
          \itemsep0em
          \item Project milestones reached.
          \item Status of previous meeting action items.
        \end{enumerate}
        \item Troubleshoot recent issues that require supervisor-specific knowledge.
        \item Determine high-level plan for project progression.
        \begin{enumerate}
          \itemsep0em
          \item Discuss problems the team expects to encounter in the near future.
          \item Identify goals the team wishes to achieve by the next meeting with the supervisor.
        \end{enumerate}
        \item Assign action items (if applicable).
        \item Discuss details of next supervisor meeting.
        \begin{enumerate}
          \itemsep0em
          \item Date \& time.
          \item Location.
        \end{enumerate}
      \end{enumerate} \\
      \hline
    \end{longtable}
  \endgroup
\textbf{Note}: All fields are subject to change depending on the supervisor’s availability or the team’s need to consult the supervisor.

\section{Team Communication Plan}
\subsection{General Purpose Communication}
The primary mode of communication for general purposes will be through Apple iMessage SMS/MMS in the team group chat.
In the event that a team member does not respond via iMessage, the secondary mode of communication will be used. This will 
be through an Instagram direct messages group chat. The primary and secondary communication channels were chosen due to their 
practicality, team members will almost certainly have their smartphone on hand, which they can use for both iMessage and Instagram 
direct messages.
A tertiary form of communication through Microsoft Teams direct messaging and/or video calls. This will be used in the event a team 
member does not respond within a reasonable time frame to primary and secondary communication methods. It will also be the main 
application through which virtual meetings will take place (both general and supervisor meetings) as it has screen sharing capabilities 
and is familiar to all team members.
\subsection{Implementation and Task Assignment}
GitHub Projects will be the primary tool used for project management related communication. A kanban board will be collectively
maintained by the team for task assignment, progress status, meeting bookkeeping, and item reviews. 
GitHub Issues will be used in tandem with the kanban board. As such, every opened issue will be required to also 
appear on the kanban board. In addition to project tasks and to-dos, issues will also be opened for every team meeting and 
every SFWRENG 4G06A/B lecture to track attendance and team contribution. There are five categories of issues that can be 
opened by the team, each having their own template: 
\begin{itemize}
  \item Lecture
  \item Meeting
  \begin{itemize}
    \item General team meetings
    \item Supervisor meetings
    \item TA meetings
  \end{itemize}
  \item Peer Review
\end{itemize}
The use of these templates standardize issue creation and streamlines project progression.


\section{Team Member Roles}
All listed team member roles are tenative and subject to change at the discretion of the team. Should role changes occur
team members should follow the process outlined below: 
\begin{enumerate}
  \item Contact all team members that are relevant to the role change.
  \item Discuss proposed changes and ensure each team member understands their new responsibilies.
  \item Notify uncontacted team members of the change and/or the overtaking of responsibilities. 
\end{enumerate}
\subsection{Project Manager}
\textbf{Assignee:} Emily Perica\\
The project manager directs the team towards successful completion of the project. 
They will be responsible for coordinating the progression by creating meeting agendas for the meeting chairperson 
to follow. They will also maintain the Kanban board and create and assign GitHub issues that concern the entire team.
\subsection{Meeting Chair}
\textbf{Assignee(s):} Jackson Lippert\\
The meeting chair is responsible for driving team meetings towards productivity and efficiency. 
This will be done by guiding and directing meetings to minimize off-topic conversation. They are required to ensure all 
meeting agenda items are addressed in a timely manner. Finally, they will facilitate constructive, collaborative
discussions during meetings.
\subsection{Meeting Minutes Recorder*}
\textbf{Assignee(s):} Ian Algenio\\
The primary responsibility of this role is to document and record team meetings. They will also track team member attendance
and address associated issues. They will record meeting discussions, issues, topics, action items, etc.
\subsection{Subject Matter Expert*}
\textbf{Assignee(s):} Mark Kogan\\
The subject matter expert is given the responsibility of maintaining a neat and organized record of source materials used 
during the lifetime of the project. This may include academic articles, open-source repositories, external libraries used, etc. 
They will also create any necessary citations for project documentation elements.
\subsection{Software Developer*}
\textbf{Assignee(s):} ALL MEMBERS\\
All team members will take on the responsibilites of this role. As developers they are responsible for designing,
testing, coding, and maintaining the software components of the project.
\subsection{Quality Assurance and Testing Specialist*}
\textbf{Assignee(s):} ALL MEMBERS\\
This role ensures the software meets stakeholder needs and quality standards. They will develop and
execute test plans and test cases, measure code coverage, and report defects in the software. Additionally,
they will verify the software meets functional and non-functional requirements such as performance, security, usability, etc.

\subsection{Music Domain Expert}
\textbf{Assignee(s):} Emily Perica\\
As the expert in the music theory domain, this person will address any confusion regarding general music theory such as terminology. They will
share their prior knowledge and experience with music whether that's playing an instrument or clarifying details about the domain.\\

\textbf{Note:} Any team member roles marked with an asterisk (*) are able to be assigned to multiple people, 
have overlapping responsibilities with other roles, and/or have responsibilities are likely to be shared amongst the enrire team.

\section{Workflow Plan}

\textbf{Issue Management Workflows}\\
When creating an issue in the Capstone Kanban Board, there are two possible workflows:
\begin{enumerate}
  \item Meeting issue is created (using the relevant template) prior to a scheduled meeting.  
  \begin{enumerate}
    \item Issue is given the 'meeting' label and assigned the 'Meeting Minutes' status.
    \item The notetaker (see section 6) will add a comment to the issue, outlining the meeting minutes and action items of each attendee.
    \item Issue is closed once all action items have been addressed.
  \end{enumerate}
  \item Project work issue is created using a blank issue.
  \begin{enumerate}
    \item Issue begins in 'To Do'
    \item One or more appropriate labels are assigned - ‘documentation’, ‘deliverable’, ‘enhancement’, ‘bug’, 'DevOps',and potentially others as the need arises.
    \item If the issue does not have the ‘deliverable’ label AND is not time sensitive, it will be moved to ‘Backlog’.
    \begin{enumerate}
      \item The issue will move out of 'Backlog' and into 'To Do' if the issue becomes time sensitive or the team has the time/resources to take it on.
    \end{enumerate}
    \item The issue moves to ‘In Progress’ once it is assigned and being worked on. It may be assigned to more than one person, depending on the scope of the issue.
    \begin{enumerate}
      \item At this point, an issue pertaining to the codebase will branch to the Git Workflow (below).
    \end{enumerate}
    \item The issue moves to ‘In Review’ once a PR has been made and approval is requested from the team.
    \item The issue moves to ‘Done’ and can be closed once all associated tasks have been completed, including all related PRs being merged/closed.
  \end{enumerate}
\end{enumerate}

In addition to the above usage of the Kanban board, all course deliverables are documented as milestones in the repository and will have issues assigned as necessary. Each milestone has a due date and a description of the tasks to be completed.

The Project Manager (see section 6 for team member roles) is responsible for making issues and keeping the project on track. As well, they will run monthly backlog grooming sessions with the team to ensure any stale issues are taken care of. \\

\noindent
\textbf{Git/GitHub Workflow}\\
A new developer on the project will begin at step 1; otherwise step 1 can be skipped:
\begin{enumerate}
  \item Developer may choose to fork or clone the repo. A fork is recommended so that contributors can display the project directly in their GitHub profile.
  \item Main is a protected branch - any commits must be made in a separate branch before being pushed to a PR. There is no specific naming scheme for branches.
  \item Commit and PR names will both start with ‘Issue \#: ’. This will be enforced through a GitHub Action.
  \begin{enumerate}
    \item Each PR will only address a single issue, but an issue may be spread out across multiple PRs.
    \item Squash commits are enforced to ensure there is a single commit per PR.
  \end{enumerate}
  \item At least one reviewer must approve the PR before it can be merged into main.\\
\end{enumerate}


\noindent
\textbf{CI/CD Workflow}\\
Actions on Push/Pull Request:
\begin{itemize}
  \item LaTeX to PDF converter
  \item Code packaged as an executable
  \item Linting
  \item Unit tests
  \item PR naming convention enforcer
\end{itemize}

\noindent Note that a PR may not be merged if any failures are present in the pipeline.

\section{Project Decomposition and Scheduling}
Link to our GitHub Project - \href{https://github.com/users/emilyperica/projects/1}{Capstone Kanban Board}.

\begin{center}
  \begin{tabular}{ |c|c| } 
    \hline
      \textbf{Due Date} & \textbf{Deliverable} \\
    \hline
      Sept 24/24 & Problem Statement, POC Plan, Development Plan \\
    \hline
      Oct 9/24 & Requirements Document Revision 0 \\
    \hline
      Oct 23/24 & Hazard Analysis \\
    \hline
      Nov 1/24 & V\&V Plan Revision 0 \\
    \hline
      Nov 11/24* & Proof of Concept Demonstration \\
    \hline
      Jan 15/25 & Design Document Revision 0 \\
    \hline
      Feb 3/25* & Revision 0 Demonstration \\
    \hline
      Mar 7/25 & V\&V Report Revision 0 \\
    \hline
      Mar 24/25* & Final Demonstration (Revision 1) \\
    \hline
      Apr \#\#/25 - TBD & Expo Demonstration \\
    \hline
      Apr 2/25 & Final Documentation (Revision 1) \\
    \hline
  \end{tabular}
\end{center}
*The demonstration should be prepared by this date, but this will not necessarily be the date of the demo itself.

Ideally, progress on a deliverable will begin at least 3 days before the previous deliverable is due (i.e., D2 work should begin on or before Sept 21). 
The above schedule will be enforced through the use of Milestones in the project repository, such that each deliverable (including its due date and all relevant information) 
is its own Milestone. The only exception to this is the Expo Demonstration, which is not expected to require any specific issues.

Each deliverable will be split into issues based on dependencies and relevant tasks. For example, when writing this document a single issue consisted of sections 7 and 8. 
They both cover project management-related plans, so completing one section provides the writer with the needed knowledge to write the other section. An alternative proposed 
method was to create an issue for each section of the document, but the granularity of this strategy may make the document feel more disjointed to the reader.

\section{Proof of Concept Demonstration Plan}

The Proof of Concept will be a demonstration of the team’s ability to develop the most crucial aspect of the final product, which is to take in a single note and output the note pitch and length. If successfully performed, it will clearly show that a viable product can be produced given enough time. 

The main risk associated with the success of the project is potential nuance introduced by human performance on instruments. Determining the time signature and differentiating between similar pitches such as \( A\sharp \) and \( B\flat \) could be potentially difficult tasks. It will become necessary to test and benchmark accuracy during audio edge cases. 

The Proof of Concept will help dispel concerns by allowing the team to show the product's potential for accurate signal processing.

Given the ability to accurately process audio, the completion of the product will be simplified, as the focus can shift towards construction of a user interface and adding features while optimizing performance and accuracy.

\section{Expected Technology}

\subsection{Tabular Overview}
\begingroup
\renewcommand{\arraystretch}{1.25}
\begin{tabular}{|>{\raggedright}p{3cm}|>{\raggedright}p{4.25cm}|>{\raggedright\arraybackslash}p{6cm}|}
  \hline
  DOMAIN & TOOL/TECHNOLOGY & DESCRIPTION \\
  \hline
  Version Control & Git, GitHub & For version control and issue tracking respectively.\\
  \hline
  Project Management & GitHub Projects & Used for Kanban board, task assignment, backlogging, etc. \\
  \hline
  Back-End Development & C/C++, MATLAB & Possible specific languages, high performance and speed is required for
  signal processesing.\\
  \hline
  Sound File Conversion & libsndfile  & A specific library that was researched and will likely be used
  for sampled sound file format conversion.\\
  \hline
  Code Formatting & Linter(s) &  Specific tools are undetermined at this stage as it is language dependent. 
  To be used with CI workflows to format code.\\
  \hline
  Testing & MATLAB unittest/automation, CMake, Unity Test, etc.  & Specific tools are undetermined, likely a unit testing framework
  that does not transcend languages (i.e. language-specific).
  \\
  \hline
  AI/ML & N/A & Will use existing libraries (pytorch, tensorflow, etc.) given the time constraint of the project, however,
  the team aims to generate our own datasets for training.\\
  \hline
  Front-End Development & HTML, CSS, JS, React Native & Languages and existing libraries deemed useful for web-based
  GUI. A framework will likely be used, in particular React Native.\\
  \hline
  UI/UX & Figma, Canva & To design and effectively prototype UI/UX elements.\\
  \hline
  Hardware & Audio Interface & An external unit for I/O of audio signals to the host device. Possible to use the host computer's 
  built-in microphone as opposed to this. \\
  \hline
  Musical Instruments & Keyboard, guitar, ukelele, etc. & Used to generate input signals.\\
  \hline
\end{tabular}
\endgroup
\subsection{Continuous Integration Plans}
GitHub Actions (GHA) will be used for continuous integration. They will employ automated workflows for various unit and integration tests
as well as code formatting/linting. One specific workflow that will be created is .tex to .pdf format conversion to ease documentation editing.
GHA will also ideally be used to test code coverage metrics. One of the major components of the project is the signal processing algorithm.  
Depending on the implementation details, different metrics in addition to the basic ones such as statement coverage might be used. For example, the designed algorithm 
might benefit from condition or path coverage for pitch detection.

\section{Coding Standard}

Our source code will adhere to the following widely adopted coding style guides:

\begin{itemize}
    \item For \textbf{C++}, which will likely be our main coding language for the signal processing, we will follow \href{https://google.github.io/styleguide/cppguide.html}{Google's C++ Style Guide} due to its widespread adoption.
    \item For \textbf{Python}, we will adhere to the \href{https://peps.python.org/pep-0008/}{PEP8 Style Guide}.
    \item For \textbf{HTML} and \textbf{CSS}, we will implement the \href{https://getbem.com/}{BEM Methodology}.
\end{itemize}

All functions, classes, and modules must be well-documented using inline comments and docstrings where appropriate. Comments should provide context when necessary, rather than stating the obvious. This ensures clear communication and explanation of code between team members, making our pull requests easy to follow and the codebase more maintainable.
Additionally, all pull requests must be reviewed by at least one other team member before being merged into the main branch. Reviews will focus on code quality, adherence to coding standards, and potential performance improvements.

\newpage{}

\section*{Appendix --- Reflection}

\input{../Reflection.tex}

\begin{enumerate}
    \item Why is it important to create a development plan prior to starting the
    project?\\
    \textbf{Emily:} The development plan that we’ve created in this document will serve as the groundwork that the rest of the project will be built off of. 
                    We’ve already seen what a difference a formalized plan can make compared with jumping in blindly - our first ideas for this project have 
                    already changed and been refined significantly since organizing our thoughts and anticipating our future needs. In creating this development 
                    plan we’ve been able to set realistic goals and discuss within the group each of our individual visions. One of the benefits of working in 
                    teams is combining different people’s thought processes and background knowledge, and yet this may also provide a disadvantage when communication 
                    is lacking. By creating a solid plan ahead of writing any code, it allows us to ensure that our levels of communication are up to the standards 
                    expected from a final year capstone project. \\ \\
    \textbf{Ian:} It ensures the team understands and thinks about the project, even if the thoughts are very abstract and lacking in detail. Without a development plan laid
                    out the parts of the project are left to interpretation for each team member, which can very significantly. One of the best examples of this was the expected technology section. 
                    Most of the specific tools and libraries we will use aren't specified, but that’s exactly the point. Even though we didn't set anything in stone like a "plan" typically would, we 
                    got to discuss possibilities, share expectations, and narrow down our choices. Without this, we would be left to deal with the consequences of not discussing expectations later down 
                    the road, when deadlines are approaching quickly and stress is high. Decisions made then would not have the opportunity to be refined like they do now, which, I think, would almost 
                    certainly impact the results of the project negatively.\\ \\
    \textbf{Jackson:} Development plans are vital to projects even outside of the realm of software projects. In my personal experience, I have been a part of group projects 
                      in school where we didn’t lay out what we were going to do and how we were going to divide the work, leading to many inefficiencies and missed requirements, resulting 
                      in loss of marks. All of that could have been prevented if we just laid out a solid development plan. Additionally, when projects come down to the deadline, wasting 
                      valuable development time on discussing what we should do leads to unnecessary stress which I have experienced. A case could be made for a situation where a development 
                      plan maybe shouldn’t be fully documented and discussed, such as a proof-of-concept that could be done within the scope of a day.\\ \\
    \textbf{Mark:} The development plan will be a crucial point of reference for the entire lifespan of our project. The construction of complex systems relies on extremely tight cohesion between collaborating partners, and failures in communication can easily result in messy and difficult situations.
                   The development plan is an objective overview of the intermediate and end goals of the product, which keeps each worker on the right track. This helps prevent potentially catastrophic miscommunications, such as different goals, different timelines, or different expectations of task delegation. 
                   In a collaborative product where communication is the difference between success and abject failure, the development plan is the starting point on which all future communication can be based.\\ \\

    \item In your opinion, what are the advantages and disadvantages of using
    CI/CD?\\
    \textbf{Emily:} CI/CD acts as the last barrier before deployment, and ensures that only well-tested code makes it to production. In our case, it will help 
                    us to maintain a uniformity across our codebase and prevent us from collecting tech debt as the project matures. That being said, it also will 
                    require a significant amount of effort to get running smoothly, both from the development side and the planning/project management side. It 
                    can also be resource intensive, which may be hard for a group of students to handle and maintain.  \\ \\
    \textbf{Ian:}  It's similar to an overall development plan in the sense that it can act as a general guideline for the project. Whether that's automation of low-level tasks
                    or testing. It's a great way to save time especially for projects with a time constraint like ours. An example would be enforcing PR conventions. Standardized and proper 
                    references to issues really helps cut down the time it takes to understand what a PR is for and why. CI/CD also has the advantage of reducing the number of errors introduced 
                    into the code base, again saving time that would be lost trying to fix these errors and conflicts. One of the disadvantages I'm worried about is how easy it is to focus too much 
                    on it. It's very new to me and I'm worried I will spend too much time trying to understand the fine details and perfecting actions rather than working on progressing the project 
                    goals themselves. I tend to do this with many things, maybe as a way of distracting myself from bigger issues. But reflecting on CI/CD helps me make sure I avoid this mistake.\\ \\
    \textbf{Jackson:} In my opinion, CI/CD can be extremely valuable in certain situations. For example, in the case of this project, it is valuable to see our merged pull requests 
                      being integrated immediately without the need for our hands-on management of deployments. This is great for saving time on a project that will last a long time because the value 
                      of CI/CD goes up the longer a project will be in its development phase. Some disadvantages could be the time it takes to set up and resolve issues that may arise because of it. 
                      This could take the form of a failed deployment, for a project which has a short development lifecycle, meaning that the time taken to set up CI/CD would be greater than the time-saving it provides.\\ \\
    \textbf{Mark:} The continuous integration pipeline is a valuable but resource-intensive element of a project. While its initial construction and continuous development require a significant time investment which is not directly related to product development, The CI pipeline helps prevent the introduction of bugs, poor design choices, or non-compliant code, resulting in a cleaner product and fewer setbacks during development. 
                   However, a downside is that more complex pipelines can take significant time to run, meaning developers might not be notified of failures until several minutes or even hours after a commit. This delay can lead to a choppy and extremely frustrating development experience. To work efficiently with a CI/CD pipeline, it’s best to simplify individual commits while working, and to have alternative tasks to complete while the pipeline runs. 
                   One of the key advantages of CI is its automatic enforcement—it triggers on every commit or pull request and ensures that merges don’t happen unless all tests pass. This ensures an ironclad adherence to rules and practices, which becomes even more critical as development teams grow. \\ \\
    
    \item What disagreements did your group have in this deliverable, if any,
    and how did you resolve them?

    We had conflicting thoughts on whether or not to mark the addition of initial content as revisions in the revision tables. This was discussed quickly through the use of requesting changes on PRs, which perfectly highlights how important a development plan is since we decided how to structure additions to our repository with PRs. 
    We also disagreed initially on how to name the project, which didn’t prove to be much of an issue as we followed standard voting protocols to find a name we were all satisfied with. There was another disagreement on whether or not to have real-time signal processing and sheet music generation as a stretch goal. 
    This problem arose when we were discussing the feasibility of the feature, and if it would even be worth adding it as a stretch goal. We resolved this with a group discussion that led to us choosing to pursue it as a stretch goal, even if it isn’t feasible, as a way to challenge ourselves.
\end{enumerate}

\newpage{}

\section*{Appendix --- Team Charter}

\subsection*{External Goals}

Our team’s primary external goal for this project is to maximize learning and skill development in software engineering. We are focused on creating a technically impressive and innovative project 
to earn the highest possible grade. Additionally, we want to ensure our project can be showcased in interviews and added to our professional portfolios and résumés. These goals will help us stand 
out when pursuing future opportunities in the tech industry.

\subsection*{Attendance}

\subsubsection*{Expectations}

Team members are expected to attend all scheduled meetings on time within reason. Leaving early is acceptable if communicated in advance and if that member has completed their contributions for 
the meeting. Additionally, all team members may leave if they feel the meeting’s agenda has been completed. Missing meetings should be a rare occurrence, and any absences must be communicated at 
least 24 hours in advance when possible.


\subsubsection*{Acceptable Excuse}

Acceptable excuses for missing a meeting or deadline include illness, family emergencies, or unavoidable academic conflicts (exams or critical project deadlines). Unacceptable excuses include 
forgetfulness, oversleeping, or personal plans that were not communicated ahead of time. All excuses must be communicated as soon as possible to the rest of the group.

\subsubsection*{In Case of Emergency}

If a team member has an emergency and cannot attend a meeting or complete their work, they should notify the team as soon as possible via the agreed-upon communication method. If possible, 
the team member should provide updates on their progress and share any work that has been completed so far, so others can step in if necessary. In case of an emergency during critical project milestones, 
the team will redistribute tasks to ensure deadlines are met.

\subsection*{Accountability and Teamwork}

\subsubsection*{Quality} 

All deliverables should be thoroughly tested, documented, and meet the project’s technical and design standards before being submitted to the team. If a team member encounters difficulties, 
they should raise a GitHub issue for discussion before the meeting to avoid delays and ensure early feedback. Pull requests raised to change code will be reviewed by at least one other team member, 
ensuring only quality code is being deployed via CI. Finally, all team members will look over and edit each others’ work for documentation, ensuring everything is clear to the reader.

\subsubsection*{Attitude}

We will foster a positive, supportive, and professional team environment where everyone’s ideas are heard and respected. During discussions, we will prioritize constructive feedback and aim to 
solve problems together. In every interaction, we will emphasize respect, inclusion, and clear communication. Any conflicts will be addressed through open discussions. If unresolved, a neutral 
team member may mediate, or the issue will be escalated to the TA.

\subsubsection*{Stay on Track}

For progress tracking we will use GitHub’s issue tracking and Kanban board features to manage tasks, ensuring that all assignments are transparent and visible to the team. Each team member 
will update the status of their tasks by moving issues across the Kanban board into their respective swim lanes. GitHub commits should reference corresponding issues, ensuring that work is traceable. 
A meeting chair is appointed to keep meetings on schedule and ensure the meeting agenda is followed.

\subsubsection*{Team Building}

To strengthen our teamwork, members are encouraged to engage with each other informally outside the scope of this project. As a team ritual, we will celebrate milestones by acknowledging 
successes and accepting every failure gracefully and as a team.

\subsubsection*{Decision Making} 

We will aim for consensus in decision-making. If a consensus cannot be reached within a set timeframe, we will hold a majority vote. For major disagreements, we will discuss the pros and 
cons of each option and if needed, consult a TA, our supervisor, or the professors of the course.

\end{document}
